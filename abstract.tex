\begin{abstract}
\thispagestyle{empty}
\noindent

(This seminar thesis aims to provide an overwiew of the concept of Inferential Models and their relation to Fiducialism, 
Frequentism, Bayesianism and imprecise Probability theory.)

\vspace{1em}

Inferential Models (IMs) are a modern framework for statistial inference. Although delivering many theoretical advantages 
over classical approaches, they are not widely used in practice. This is mainly due to the fact that they might seem 
unapproachable to many readers, who are not familiar with imprecise probability thoery and the concept of Data Generating
Equations (DGE). Therefore this seminar thesis aims to provide a gentle introduction to the concept of IM, which may be 
more accessible to a wider audience.

\vspace{1em}

Inferential Models were developed as a framework for prior-free posterior probabilistic inference. They embrace 
non-additive degrees of belief and aim to provide \textit{valid}, data-based inferences. 

\end{abstract}