\usepackage[style=numeric]{biblatex}
\addbibresource{Sources.bib}

\usepackage{enumitem}


\usepackage{geometry}
\geometry{a4paper,top=35mm,bottom=30mm,textwidth=160mm}
% \usepackage{chngcntr}

\renewcommand{\thechapter}{\Roman{chapter}}

\usepackage{tocloft}

\usepackage{amsthm}
\theoremstyle{definition}
\newtheorem{definition}{Definition}[section]
\newtheorem{example}{Example}[section]

\renewcommand{\cftchapnumwidth}{2em}
\renewcommand{\cftsecindent}{2em}
\renewcommand{\cftsubsecindent}{5em}
\setlength{\cftbeforechapskip}{1em} % Kapitel-Abstand im TOC
\setlength{\cftbeforesecskip}{0.5em}
\renewcommand{\thesection}{\arabic{section}}
\renewcommand{\thesubsection}{\thesection.\arabic{subsection}}
\renewcommand{\thesubsubsection}{\thesubsection.\arabic{subsubsection}}
%\setcounter{chapter}{-1}
\counterwithout{section}{chapter}

\usepackage{tikz}
\usetikzlibrary{bayesnet}
\tikzset{
      dot hidden/.style={},
      line hidden/.style={},
      dot colour/.style={dot hidden/.append style={color=#1}},
      dot colour/.default=black,
      line colour/.style={line hidden/.append style={color=#1}},
      line colour/.default=black
    }
\NewDocumentCommand{\drawdie}{O{}m}{%
      \begin{tikzpicture}[x=1em,y=1em,radius=0.1,#1]
        \draw[rounded corners=0.5,line hidden] (0,0) rectangle (1,1);
        \ifodd #2
          \fill[dot hidden] (0.5,0.5) circle;
        \fi
        \ifnum #2>1
          \fill[dot hidden] (0.2,0.2) circle;
          \fill[dot hidden] (0.8,0.8) circle;
        \fi
        \ifnum #2>3
          \fill[dot hidden] (0.2,0.8) circle;
          \fill[dot hidden] (0.8,0.2) circle;
        \fi
        \ifnum #2>5
          \fill[dot hidden] (0.8,0.5) circle;
          \fill[dot hidden] (0.2,0.5) circle;
        \fi
      \end{tikzpicture}%
    }


